Com o avanço das tecnologias de RV e Realidade Aumentada (RA), diversas soluções educacionais têm sido desenvolvidas com o intuito de tornar o processo de aprendizagem mais interativo e imersivo. Essas tecnologias têm sido particularmente eficazes na educação técnica, proporcionando ambientes simulados onde os estudantes podem realizar práticas e experimentos sem a necessidade de recursos físicos. No campo da infraestrutura de computadores, algumas iniciativas têm explorado o uso de ambientes virtuais para simular laboratórios e práticas técnicas, permitindo que os alunos se envolvam ativamente com o conteúdo, porém, muitas dessas soluções dependem do uso de equipamentos especializados, como headsets de realidade virtual (por exemplo, Oculus Rift ou HTC Vive), o que torna o acesso a essas tecnologias restrito, principalmente em contextos educacionais com menor disponibilidade de investimento.

Maroon é um laboratório virtual projetado para expêriencias de aprendizado com configurações de RV, o projeto de \cite{PIRKER2021}, foi projetado para o entendimento e a melhor compreensão da física. Maroon foi criado com o motor de jogos $Unity3D^{4}$ e projetou de forma semelhante a um jogo em primeira pessoa. A versão principal do Maroon suporta diferentes formas de interação, como o uso convencional de um mouse e teclado, e por outro lado que seria o principal, com dispositivos de RV, como o HTC Vive, o Oculos Rift, ou com RV móveis (com o próprio smartphone). Em um primeiro estudo, a experiência foi avaliada por 14 futuros professores de física para identificar e discutir o potencial do laboratório em sala de aula. Neste estudo verificou-se que a aplicação é envolvente e promissora, entretanto, no estado atual os professores optam por utiliza-lo em "dias de projetos" dedicados, em vez de fazer uso regular da tecnologia. A observação qualitativa e o feedback que tiveram dos participantes também indicam preocupações de que o uso de RV em escala de sala com grandes grupos de estudantes (por exemplo, 20 a 30 estudantes) possa ser esmagador e que os estudantes possam começar a usar a tecnologia de forma inadequada, em vez de se concentrar no conteúdo de aprendizado.

No estudo de \cite{SCHMID2022}, foi criado um ambiente de aprendizagem em RV tridimensional para o Microsoft HoloLens para o aprendizado do conceitos de campo elétrico e potencial elétrico. Este sistema foi desenvolvido utilizando o motor de jogos Unity3D. Este consiste em 18 tarefas de múltipla escolha. Cada tarefa mostra uma paisagem de um potencial elétrico. Para cada paisagem, o campo vetorial correspondente deve ser identificado entre seis opções fornecidas. Ao clicar em um campo vetorial, o ambiente de aprendizado em RV indica diretamente se a seleção está correta ou não. As paisagens 3D são ilustradas em azul. Cada paisagem representa, na verdade, um gráfico de um potencial 2D, e o eixo vertical mostra a magnitude do potencial ao longo dos eixos horizontais x e y. Locais onde o potencial é alto são ilustrados por colinas, mas também por uma faceta mais clara da cor azul. Os estudantes podem tentar três vezes para cada paisagem identificar o campo vetorial correspondente. Eles obtêm pontos quando selecionam a solução correta. Com o headset de RV, eles podem caminhar ao redor e observar a paisagem do potencial de todos os lados. "Considerando nossa primeira questão de pesquisa, os resultados mostram que a maioria dos estudantes apresentou um progresso substancial na aprendizagem ao utilizar o ambiente de aprendizagem em RV por 15 minutos. Isso indica que a realidade virtual pode, de fato, ser uma tecnologia adequada para ajudar os estudantes a compreender a relação entre o campo elétrico e o potencial." \cite{SCHMID2022}. A maioria das limitações do projeto são relacionadas às paisagens 3D dos potenciais. Seria muito difícil compreender a distribuição do potencial de uma só vez. Outro problema é que é muito difícil ilustrar proporcionalidade com cores ou luminosidade. Com nosso ambiente de aprendizado em RV, tentamos avançar um passo na representação da física invisível no mundo 3D real, mas ainda enfrentamos o problema de não conseguir ilustrar o mundo como ele é.

Agora com um projeto mais famoso, o \textit{Google Expeditions} é uma poderosa plataforma educacional que permite que alunos e professores utilizem RA e RV em suas expêriencias. Em seu modo de RV, os alunos são posicionados em lugares famosos e dificeis de visitar fisicamente - seja uma viagem ao fundo do oceano, uma visita aos dinossauros ou a exploração de civilizações históricas. "Embora o \textit{Google Expeditions} represente um avanço significativo na integração de tecnologias imersivas à educação, a sua adoção em larga escala ainda esbarra em entraves estruturais. A exigência por dispositivos móveis atualizados, óculos de RV e conexão estável com a internet pode representar um grande obstáculo, especialmente em escolas da rede pública e em regiões com acesso tecnológico limitado. A ausência de ambientes adequados para a expêriencia imersiva, como salas amplas e com pouca interferência sonora, também é uma questão a ser considerada." \cite{TERRA2025}.

De modo geral, os projetos analisados, Maroon, o sistema de \cite{SCHMID2022} e o \textit{Google Expeditions}, demonstram o potencial das tecnologias imersivas para o aprendizado, mas compartilham limitações significativas, sobretudo o alto custo dos equipamentos e a dependência de infraestrutura avançada. Nesse contexto, o SAVIT se destaca por propor uma solução de baixo custo e acessível, utilizando apenas o smartphone e a câmera do notebook para criar um ambiente de realidade virtual imersivo. Diferente das demais iniciativas, o SAVIT é voltado ao ensino técnico prático, permitindo que o estudante aprenda, de forma interativa e gamificada, como montar computadores e estruturar redes de forma completa. Essa abordagem alia acessibilidade, aplicabilidade real e engajamento, democratizando o acesso ao ensino tecnológico e promovendo uma aprendizagem mais significativa e inclusiva.