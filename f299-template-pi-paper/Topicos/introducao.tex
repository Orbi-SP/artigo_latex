O avanço acelerado da tecnologia tem remodelado praticamente todos os aspectos da sociedade, incluindo a educação, que se vê desafiada a acompanhar essas mudanças de forma eficiente e inclusiva. Em um cenário onde a inovação tecnológica é fundamental para a preparação dos estudantes para o mercado de trabalho, é essencial que todos tenham acesso a ferramentas de ensino que possibilitem um aprendizado mais intuitivo, prático e eficaz \cite{MORAN2018}. No entanto, em muitas instituições educacionais, especialmente em comunidades com recursos financeiros limitados, a falta de infraestrutura e de equipamentos adequados impede que os alunos adquiram as habilidades necessárias para se destacarem em áreas de maior conhecimento tecnológico, como a montagem de hardware, programação e redes de computadores \cite{UNESCO2021}.

Com base nos dados coletados por meio de um questionário aplicado a cinquenta alunos de cursos relacionados a Sistemas Operacionais e Redes de Computadores, foi possível observar que a disponibilidade de equipamentos adequados para a realização de aulas práticas ainda é limitada em muitas instituições. Apenas 26\% dos respondentes afirmaram que sempre tiveram acesso a recursos físicos, enquanto 58\% relataram que essa disponibilidade ocorria na maioria das vezes, e 16\% afirmaram que raramente ou nunca tiveram contato com o equipamento necessário.

Neste contexto, este artigo tem como objetivo apresentar um sistema educacional baseado em Realidade Virtual (RV), o SAVIT (Sistema de Aprendizado Virtual para Infraestrutura Tecnológica). Nossa aplicação surge como uma solução acessível, moderna e eficaz para o ensino de disciplinas técnicas, permitindo a simulação real de montagem de um simples computador até a instalação de uma infraestrutura de redes completa. O foco é tornar atividades complexas e detalhadas em algo simples e memorável para os estudantes, que poderão aprender o termo e a função de cada componente em um ambiente virtual imersivo — superando a barreira do custo elevado de materiais físicos e equipamentos de laboratório \cite{FREITAS2020}.
