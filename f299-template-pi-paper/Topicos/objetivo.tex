O objetivo desta aplicação é proporcionar aos estudantes um ambiente de treinamento virtual e imersivo com o uso de RV, que possa ser utilizado tanto em contextos educacionais quanto domiciliares. O sistema foi concebido para ser de fácil acesso e operação, dispensando a necessidade de equipamentos de alto desempenho computacional.

A aplicação apresenta uma interface inicial que permite ao usuário realizar configurações de acessibilidade conforme suas necessidades. Após essa etapa, o participante é conduzido a um ambiente tridimensional no qual lhe é apresentada uma máquina virtual. A tarefa consiste em realizar a montagem e o funcionamento correto desse equipamento, simulando situações práticas de aprendizagem voltadas ao estudo de hardware e redes.

Durante todo o processo, o estudante conta com o suporte de um mascote virtual, projetado para atuar como assistente pedagógico. Esse personagem tem a função de orientar o usuário, explicando a nomenclatura e a importância de cada componente, bem como instruindo-o quanto à execução dos comandos necessários para a realização das ações no ambiente virtual.