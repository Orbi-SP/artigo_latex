\documentclass[
  a4paper,
  12pt,
  english,
  brazilian,
]{article}

\usepackage{pdflscape}
\usepackage{pdfpages}
\usepackage[]{fatec-article}
\usepackage{float}

\Author{1}{Name={Éric Pinheiro Abrahão Leandro \\ Kaique Silva Pinto\\ Lucas do Nascimento Miura\\ Luiz Ricardo Wandenkolk Medina\\ Pedro Henrique da Silva}}

\Author{2}{Name={\{Eric.leandro@fatec.sp.gov.br\} \\ \{kaique.silva58@fatec.sp.gov.br\} \\ \{lucas.miura@fatec.sp.gov.br\} \\ \{luiz.medina@fatec.sp.gov.br\} \\ \{pedro.silva373@fatec.sp.gov.br\}}}

\Keyword{1}{Sistema Educacional}{Educational system}
\Keyword{2}{Realidade Virtual}{Virtual Reality}
\Keyword{3}{Redes de Computadores}{Computer Networks}

\begin{Abstract}[brazilian]
  Este trabalho apresenta uma proposta de sistema educacional acessível, baseado em Realidade Virtual (RV), voltado ao ensino prático de montagem de hardware e redes de computadores. A ideia central é permitir que estudantes, mesmo em instituições com poucos recursos, possam vivenciar a prática por meio de simulações imersivas e interativas, reduzindo a dependência de equipamentos físicos, muitas vezes caros e de difícil acesso. Ao recriar ambientes realistas em um formato virtual, o projeto busca facilitar o aprendizado técnico, estimular a autonomia do aluno e ampliar as possibilidades de ensino-aprendizagem na área de Tecnologia da Informação (TI). Além disso, a proposta considera a inclusão digital como um dos seus pilares, contribuindo para uma educação mais democrática, moderna e conectada com as demandas do mercado.
\end{Abstract}

\begin{Abstract}[english]
  This paper presents a proposal for an accessible educational system based on Virtual Reality (VR), aimed at the practical teaching of hardware assembly and computer networks. The central idea is to enable students, even in institutions with limited resources, to experience hands-on practice through immersive and interactive simulations, reducing dependence on physical equipment, which is often expensive and hard to access. By recreating realistic environments in a virtual format, the project aims to facilitate technical learning, encourage student autonomy, and expand teaching-learning possibilities in the field of Information Technology (IT). Furthermore, the proposal considers digital inclusion as one of its core principles, contributing to a more democratic, modern, and market-connected education.
\end{Abstract}

\makeindex

\addbibresource{fatec-article.bib}

\begin{document}

\section*{Introdução}%
\label{sect:intro}
O avanço acelerado da tecnologia tem remodelado praticamente todos os aspectos da sociedade, incluindo a educação, que se vê desafiada a acompanhar essas mudanças de forma eficiente e inclusiva. Em um cenário onde a inovação tecnológica é fundamental para a preparação dos estudantes para o mercado de trabalho, é essencial que todos tenham acesso a ferramentas de ensino que possibilitem um aprendizado mais intuitivo, prático e eficaz \cite{MORAN2018}. No entanto, em muitas instituições educacionais, especialmente em comunidades com recursos financeiros limitados, a falta de infraestrutura e de equipamentos adequados impede que os alunos adquiram as habilidades necessárias para se destacarem em áreas de maior conhecimento tecnológico, como a montagem de hardware, programação e redes de computadores \cite{UNESCO2021}.

Com base nos dados coletados por meio de um questionário aplicado a cinquenta alunos de cursos relacionados a Sistemas Operacionais e Redes de Computadores, foi possível observar que a disponibilidade de equipamentos adequados para a realização de aulas práticas ainda é limitada em muitas instituições. Apenas 26\% dos respondentes afirmaram que sempre tiveram acesso a recursos físicos, enquanto 58\% relataram que essa disponibilidade ocorria na maioria das vezes, e 16\% afirmaram que raramente ou nunca tiveram contato com o equipamento necessário.

Neste contexto, este artigo tem como objetivo apresentar um sistema educacional baseado em Realidade Virtual (RV), o SAVIT (Sistema de Aprendizado Virtual para Infraestrutura Tecnológica). Nossa aplicação surge como uma solução acessível, moderna e eficaz para o ensino de disciplinas técnicas, permitindo a simulação real de montagem de um simples computador até a instalação de uma infraestrutura de redes completa. O foco é tornar atividades complexas e detalhadas em algo simples e memorável para os estudantes, que poderão aprender o termo e a função de cada componente em um ambiente virtual imersivo — superando a barreira do custo elevado de materiais físicos e equipamentos de laboratório \cite{FREITAS2020}.


\section*{OBJETIVO}\label{sect:obj}

O objetivo desta aplicação é proporcionar aos estudantes um ambiente de treinamento virtual e imersivo com o uso de RV, que possa ser utilizado tanto em contextos educacionais quanto domiciliares. O sistema foi concebido para ser de fácil acesso e operação, dispensando a necessidade de equipamentos de alto desempenho computacional.

A aplicação apresenta uma interface inicial que permite ao usuário realizar configurações de acessibilidade conforme suas necessidades. Após essa etapa, o participante é conduzido a um ambiente tridimensional no qual lhe é apresentada uma máquina virtual. A tarefa consiste em realizar a montagem e o funcionamento correto desse equipamento, simulando situações práticas de aprendizagem voltadas ao estudo de hardware e redes.

Durante todo o processo, o estudante conta com o suporte de um mascote virtual, projetado para atuar como assistente pedagógico. Esse personagem tem a função de orientar o usuário, explicando a nomenclatura e a importância de cada componente, bem como instruindo-o quanto à execução dos comandos necessários para a realização das ações no ambiente virtual.

\section*{ESTADO DA ARTE}\label{sect:estadoarte}

Com o avanço das tecnologias de RV e Realidade Aumentada (RA), diversas soluções educacionais têm sido desenvolvidas com o intuito de tornar o processo de aprendizagem mais interativo e imersivo. Essas tecnologias têm sido particularmente eficazes na educação técnica, proporcionando ambientes simulados onde os estudantes podem realizar práticas e experimentos sem a necessidade de recursos físicos. No campo da infraestrutura de computadores, algumas iniciativas têm explorado o uso de ambientes virtuais para simular laboratórios e práticas técnicas, permitindo que os alunos se envolvam ativamente com o conteúdo, porém, muitas dessas soluções dependem do uso de equipamentos especializados, como headsets de realidade virtual (por exemplo, Oculus Rift ou HTC Vive), o que torna o acesso a essas tecnologias restrito, principalmente em contextos educacionais com menor disponibilidade de investimento.

Maroon é um laboratório virtual projetado para expêriencias de aprendizado com configurações de RV, o projeto de \cite{PIRKER2021}, foi projetado para o entendimento e a melhor compreensão da física. Maroon foi criado com o motor de jogos $Unity3D^{4}$ e projetou de forma semelhante a um jogo em primeira pessoa. A versão principal do Maroon suporta diferentes formas de interação, como o uso convencional de um mouse e teclado, e por outro lado que seria o principal, com dispositivos de RV, como o HTC Vive, o Oculos Rift, ou com RV móveis (com o próprio smartphone). Em um primeiro estudo, a experiência foi avaliada por 14 futuros professores de física para identificar e discutir o potencial do laboratório em sala de aula. Neste estudo verificou-se que a aplicação é envolvente e promissora, entretanto, no estado atual os professores optam por utiliza-lo em "dias de projetos" dedicados, em vez de fazer uso regular da tecnologia. A observação qualitativa e o feedback que tiveram dos participantes também indicam preocupações de que o uso de RV em escala de sala com grandes grupos de estudantes (por exemplo, 20 a 30 estudantes) possa ser esmagador e que os estudantes possam começar a usar a tecnologia de forma inadequada, em vez de se concentrar no conteúdo de aprendizado.

No estudo de \cite{SCHMID2022}, foi criado um ambiente de aprendizagem em RV tridimensional para o Microsoft HoloLens para o aprendizado do conceitos de campo elétrico e potencial elétrico. Este sistema foi desenvolvido utilizando o motor de jogos Unity3D. Este consiste em 18 tarefas de múltipla escolha. Cada tarefa mostra uma paisagem de um potencial elétrico. Para cada paisagem, o campo vetorial correspondente deve ser identificado entre seis opções fornecidas. Ao clicar em um campo vetorial, o ambiente de aprendizado em RV indica diretamente se a seleção está correta ou não. As paisagens 3D são ilustradas em azul. Cada paisagem representa, na verdade, um gráfico de um potencial 2D, e o eixo vertical mostra a magnitude do potencial ao longo dos eixos horizontais x e y. Locais onde o potencial é alto são ilustrados por colinas, mas também por uma faceta mais clara da cor azul. Os estudantes podem tentar três vezes para cada paisagem identificar o campo vetorial correspondente. Eles obtêm pontos quando selecionam a solução correta. Com o headset de RV, eles podem caminhar ao redor e observar a paisagem do potencial de todos os lados. "Considerando nossa primeira questão de pesquisa, os resultados mostram que a maioria dos estudantes apresentou um progresso substancial na aprendizagem ao utilizar o ambiente de aprendizagem em RV por 15 minutos. Isso indica que a realidade virtual pode, de fato, ser uma tecnologia adequada para ajudar os estudantes a compreender a relação entre o campo elétrico e o potencial." \cite{SCHMID2022}. A maioria das limitações do projeto são relacionadas às paisagens 3D dos potenciais. Seria muito difícil compreender a distribuição do potencial de uma só vez. Outro problema é que é muito difícil ilustrar proporcionalidade com cores ou luminosidade. Com nosso ambiente de aprendizado em RV, tentamos avançar um passo na representação da física invisível no mundo 3D real, mas ainda enfrentamos o problema de não conseguir ilustrar o mundo como ele é.

Agora com um projeto mais famoso, o \textit{Google Expeditions} é uma poderosa plataforma educacional que permite que alunos e professores utilizem RA e RV em suas expêriencias. Em seu modo de RV, os alunos são posicionados em lugares famosos e dificeis de visitar fisicamente - seja uma viagem ao fundo do oceano, uma visita aos dinossauros ou a exploração de civilizações históricas. "Embora o \textit{Google Expeditions} represente um avanço significativo na integração de tecnologias imersivas à educação, a sua adoção em larga escala ainda esbarra em entraves estruturais. A exigência por dispositivos móveis atualizados, óculos de RV e conexão estável com a internet pode representar um grande obstáculo, especialmente em escolas da rede pública e em regiões com acesso tecnológico limitado. A ausência de ambientes adequados para a expêriencia imersiva, como salas amplas e com pouca interferência sonora, também é uma questão a ser considerada." \cite{TERRA2025}.

De modo geral, os projetos analisados, Maroon, o sistema de \cite{SCHMID2022} e o \textit{Google Expeditions}, demonstram o potencial das tecnologias imersivas para o aprendizado, mas compartilham limitações significativas, sobretudo o alto custo dos equipamentos e a dependência de infraestrutura avançada. Nesse contexto, o SAVIT se destaca por propor uma solução de baixo custo e acessível, utilizando apenas o smartphone e a câmera do notebook para criar um ambiente de realidade virtual imersivo. Diferente das demais iniciativas, o SAVIT é voltado ao ensino técnico prático, permitindo que o estudante aprenda, de forma interativa e gamificada, como montar computadores e estruturar redes de forma completa. Essa abordagem alia acessibilidade, aplicabilidade real e engajamento, democratizando o acesso ao ensino tecnológico e promovendo uma aprendizagem mais significativa e inclusiva.

\section*{METODOLOGIA}\label{sect:metodologia}

O desenvolvimento do projeto SAVIT foi estruturado com base em uma abordagem modular, que permitiu a organização das tarefas entre os membros da equipe de forma eficiente e colaborativa. Cada módulo do sistema corresponde a uma etapa específica do processo de construção da aplicação, sendo fundamentado em ferramentas amplamente utilizadas no desenvolvimento de soluções baseadas em realidade virtual e aumentada (RV/RA).

O SAVIT será desenvolvido em C\# utilizando o motor gráfico Unity3D. Ao abrir a aplicação, o usuário encontrará um menu inicial com opções para ajustar configurações gráficas, de jogabilidade e de acessibilidade. Ao clicar em “Jogar”, a experiência será iniciada. Nesta primeira versão, o SAVIT será mais simples, oferecendo uma simulação interativa em 3D do processo de montagem de um computador, onde o usuário poderá, por exemplo, posicionar a placa-mãe dentro do gabinete. Abaixo está o fluxograma para ilustrar melhor o funcionamento:

\begin{figure}[H]
    \centering
    \includegraphics[width=0.9\textwidth]{Imagem/Fluxograma_SAVIT.png}
    \caption{Fluxograma do funcionamento do sistema SAVIT.} 
    \label{fig:fluxo_SAVIT}
\end{figure}

Como representado pelo fluxograma acima, após fazer um gesto de segurar com as mãos o sistema irá levantar a peça significando que o item foi pego, como na Imagem A e B.

\begin{figure}[H]
    \centering
    \captionsetup{labelformat=empty}
    \includegraphics[width=0.9\textwidth]{Imagem/imagem A.png}
    \caption{Imagem A - O item sobre a mesa será coletado após fazer um gesto de "pegar" com as mãos.} 
    \label{fig:placa_mesa}
\end{figure}

\begin{figure}[H]
    \centering
    \captionsetup{labelformat=empty}
    \includegraphics[width=0.9\textwidth]{Imagem/imagem B.png}
    \caption{Imagem B - O item está sendo "segurado" pelo usuário.} 
    \label{fig:placa_ar}
\end{figure}

Seguindo o fluxograma após mover a peça ao local correto, ela será alocada e travada, como na Imagem C.

\begin{figure}[H]
    \centering
    \captionsetup{labelformat=empty}
    \includegraphics[width=0.9\textwidth]{Imagem/imagem C.png}
    \caption{Imagem C - A peça foi alocada corretamente assim travando-a no gabinete.} 
    \label{fig:placa_travada}
\end{figure}



\section*{REFERÊNCIAS}\label{sect:referencias}

\input{Topicos/referencias}

\newpage

\printbibliography

\end{document}