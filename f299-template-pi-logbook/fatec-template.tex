\documentclass[
landscape,
  a4paper,
  12pt,
  english,
  brazilian,
]{article}

\usepackage[]{fatec-article}
\usepackage{setspace}

\begin{document}

\section*{Instruções para o preenchimento}
\doublespacing
\begin{enumerate}
    \item O Diário de Bordo é usado para registrar atividades, progressos, ideias e desafios enfrentados em um projeto ou durante a rotina de trabalho. Serve como um registro cronológico e detalhado das operações diárias, facilitando a organização e o acompanhamento das tarefas.
    \doublespacing
    \item Durante o registro das atividades deve-se incluir detalhes como datas, horários, descrições de tarefas, nomes de participantes e observações relevantes.  Esta documentação contínua ajuda na avaliação do progresso de projetos ou atividades, permitindo ajustes e melhorias contínuas nos processos.
    \doublespacing
    \item Para evidenciar a realização das tarefas, você poderá utilizar a criação de anexos para adicionar anotações, fotos, prints, questionários, entre outros.
\end{enumerate}

 \begin{table}[]
\centering


\begin{tabular}{|>{\raggedright\arraybackslash}p{0.2\linewidth}|l|l|>{\raggedright\arraybackslash}p{0.2\linewidth}|>{\raggedright\arraybackslash}p{0.2\linewidth}|}

\hline

Inscrição para o GrandPrix do SENAC& 10/10/2024& 10/10/2024& João Kusaka; Matheus Abrahão; Tiago Rodrigues; Victor Roder; Isabele Queiroz. &Foi decidido pela maioria dos membros em participar do GrandPrix do Senac utilizando o tema do nosso P.I.\\\hline                             

\end{tabular}
\end{table}

\end{document}