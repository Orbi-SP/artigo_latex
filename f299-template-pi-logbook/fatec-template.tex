\documentclass[
landscape,
  a4paper,
  12pt,
  english,
  brazilian,
]{article}

\usepackage[]{fatec-article}
\usepackage{setspace}

\begin{document}

\section*{Instruções para o preenchimento}
\doublespacing
\begin{enumerate}
    \item O Diário de Bordo é usado para registrar atividades, progressos, ideias e desafios enfrentados em um projeto ou durante a rotina de trabalho. Serve como um registro cronológico e detalhado das operações diárias, facilitando a organização e o acompanhamento das tarefas.
    \doublespacing
    \item Durante o registro das atividades deve-se incluir detalhes como datas, horários, descrições de tarefas, nomes de participantes e observações relevantes.  Esta documentação contínua ajuda na avaliação do progresso de projetos ou atividades, permitindo ajustes e melhorias contínuas nos processos.
    \doublespacing
    \item Para evidenciar a realização das tarefas, você poderá utilizar a criação de anexos para adicionar anotações, fotos, prints, questionários, entre outros.
\end{enumerate}

 \begin{table}[]
\centering


\begin{tabular}{|>{\raggedright\arraybackslash}p{0.2\linewidth}|l|l|>{\raggedright\arraybackslash}p{0.2\linewidth}|>{\raggedright\arraybackslash}p{0.2\linewidth}|}

\hline


Integração de um novo integrante& 10/08/2025& 10/08/2025& Lucas Miura, Kaique Silva, Eric Pinheiro, Pedro Henrique, Luiz Medina &Entrada do novo membro Luiz Ricardo na equipe ORBI.\\\hline   
Ínicio das atividades do primeiro semestre & 13/08/2025 & 13/08/2025 & Lucas Miura, Kaique Silva, Eric Pinheiro, Pedro Henrique, Luiz Medina & Organização do tempo do projeto integrador da equipe e formulação de novas tarefas, discussões sobre a entrada de um novo membro\\\hline 
Pesquisa de Campo& 10/09/2025 & 29/09/2025 & Lucas Miura, Kaique Silva, Eric Pinheiro & Criação da pesquisa, análise das questões, pesquisa em campo, análise dos dados coletados.\\\hline
Diagrama de Banco de Dados& 08/09/2025 & 28/10/2025 & Lucas Miura & Releitura, análise e criação do diagrama de banco de dados não relacional.\\\hline                         


\end{tabular}
\end{table}






 \begin{table}[]
\centering


\begin{tabular}{|>{\raggedright\arraybackslash}p{0.2\linewidth}|l|l|>{\raggedright\arraybackslash}p{0.2\linewidth}|>{\raggedright\arraybackslash}p{0.2\linewidth}|}

\hline


Criação da Cena "Gabinete" com placa-mãe& 07/10/2025& 18/10/2025& Luiz Medina &Cena que permite posicionar a placa-mãe no gabinete usando gestos de mão reconhecidos pela câmera (movimento lateral/encaixe)\\\hline   
Criação do vídeo do pitch& 13/10/2025& 28/10/2025& Lucas Miura, Eric Pinheiro &Criação do vídeo do Pitch apresentando a problemática e como nosso projeto pode resolver\\\hline 
Gravação para o pitch& 13/10/2025& 24/10/2025& Lucas Miura, Kaique Silva, Eric Pinheiro, Pedro Henrique, Luiz Medina &Gravação da narração para utilizar no pitch\\\hline 
Junção da cena da memória RAM com a cena do gabinete& 19/10/2025& 27/10/2025& Luiz Medina &Unificação da cena do gabinete com a da placa-mãe, substituição de modelo antigo e integração do fluxo de montagem\\\hline 
Interface do website& 20/10/2025& 28/10/2025& Lucas Miura &Análise e justificativa da não aplicação de uma interface do website\\\hline
Banner& 20/10/2025& 28/10/2025& Kaique Silva &Alterações conforme atualizações do projeto e artigo científico\\\hline 
Artígo científico& 12/10/2025& 28/10/2025& Kaique Silva &Criação do artigo cientifico\\\hline  
Canvas& 20/10/2025& 28/10/2025& Lucas Miura &Análise, atualização e reaproveitamento do modelo de negócios\\\hline  

\end{tabular}
\end{table}




 \begin{table}[]
\centering


\begin{tabular}{|>{\raggedright\arraybackslash}p{0.2\linewidth}|l|l|>{\raggedright\arraybackslash}p{0.2\linewidth}|>{\raggedright\arraybackslash}p{0.2\linewidth}|}

\hline


Diagrama de Casos de uso, classe e objetos& 20/10/2025& 27/10/2025& Lucas Miura &Atualização e revisão da análise do diagrama de caso de uso, classes e objetos\\\hline
Diagrama e especificações da infraestrutura de rede& 20/10/2025& 28/10/2025& Lucas Miura &Análise, atualização e reaproveitamento do diagrama da infraestrutura de rede\\\hline
UI de alta fidelidade& 20/10/2025& 28/10/2025& Lucas Miura &Análise e justificativa da não aplicação da UI de Alta Fidelidade\\\hline
Atualização da análise SWOT& 27/10/2025& 28/10/2025& Lucas Miura, Éric Pinheiro &Atualização e revisão da análise SWOT\\\hline
Organização dos artefatos e documentações para entrega& 28/10/2025& 29/10/2025& Lucas Miura &Conferência e entrega das documentações do projeto\\\hline


\end{tabular}
\end{table}

\end{document}